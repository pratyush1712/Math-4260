\documentclass[11pt,onecolumn]{article}
\usepackage{amssymb, amsmath, amsthm,graphicx, paralist,algpseudocode,algorithm,cancel,url,color}
\usepackage{sectsty}
\usepackage{fancyvrb}
\usepackage{mathrsfs}
\usepackage{multirow}
\usepackage{hhline}
\usepackage{booktabs}
\usepackage[table]{xcolor}
\usepackage{tikz}
% \usepackage[framed,numbered,autolinebreaks,useliterate]{mcode}
\usepackage{listings}
\usepackage{enumitem}

\newcommand{\bvec}[1]{\mathbf{#1}}
\newcommand{\R}{\mathbb{R}}
\newcommand{\C}{\mathbb{C}}
\newcommand{\Rn}{\R^{n\times n}}
\newcommand{\Rmn}{\R^{m\times n}}
\newcommand{\Cn}{\C^{n\times n}}
\newcommand{\Cmn}{\C^{m\times n}}
\newcommand{\cO}{\mathcal{O}}
\DeclareMathOperator{\Tr}{Tr}
\DeclareMathOperator{\trace}{trace}
\DeclareMathOperator{\diag}{diag}
\DeclareMathOperator{\vspan}{span}
\sectionfont{\Large\sc}
\subsectionfont{\sc}
\usepackage[margin=1 in]{geometry}
\begin{document}
\noindent
\textsc{\Large Numerical analysis: Homework 3}\\
Instructor: Anil Damle\\
Due: March 8, 2024\\

\subsection*{Policies}
You may discuss the homework problems freely with other students, but please
refrain from looking at their code or writeups (or sharing your own).
Ultimately, you must implement your own code and write up your own solution to
be turned in. Your solution, including plots and requested output from your code
should be typeset and submitted via the Gradescope as a pdf file. This file must be self contained for grading. Additionally, please
submit any code written for the assignment as zip
file to the separate Gradescope assignment for code.\\
\hrule
\vspace{1cm}
\noindent

\subsection*{Question 1:}
Assume that we are given $A\in\mathbb{R}^{n\times n},$ $A=A^T,$ and $A$ has eigenvalue and vector pairs $\left\{(v_i,\lambda_i)\right\}_{i=1}^n.$ Furthermore, assume that $\lvert\lambda_1\rvert = \lvert\lambda_2\rvert > \lvert \lambda_3\rvert \geq \lvert\lambda_4\rvert \geq \cdots.$ 

\begin{enumerate}[label=(\alph*)]
\item Prove that for any initial guess $v^{(0)}$ such that $v^{(0)}$ is not simultaneously orthogonal to both $v_1$ and $v_2$ the power method yields iterates $v^{(k)}$ that converge to lie in the span of $v_1$ and $v_2.$
\item What is the rate of convergence of $$\left(1-\left\|\left(v^{(k)}\right)^T\begin{bmatrix}v_1 & v_2\end{bmatrix}\right\|^2_2\right)^{1/2}?$$ 
\item Does the associated eigenvalue estimate via the Rayleigh quotient necessarily converge in this setting? what about if $\lambda_1 = \lambda_2?$
\end{enumerate}

\subsection*{Question 2:}
Assume that we are given $A\in\mathbb{R}^{n\times n},$ $A=A^T,$ and $A$ has eigenvalue and vector pairs $\left\{(v_i,\lambda_i)\right\}_{i=1}^n.$ Furthermore, assume that $\lambda_1 \geq \lambda_2 \geq \cdots \geq \lambda_\ell > \lambda_{\ell+1}\geq \cdots \geq \lambda_n \geq 0.$ Now, say we run simultaneous iteration (also known an orthogonal iteration) to compute $\vspan\{v_1,\ldots,v_\ell\}$ and the associated eigenvalues $\lambda_1,\ldots,\lambda_\ell.$

\begin{enumerate}[label=(\alph*)]
\item If we denote $\hat{\lambda}_1^{(k)}$ as our guess for $\lambda_1$ at iteration $k,$ show that $\lambda_1 \geq \hat{\lambda}_1^{(k)}$ for all $k.$ I.e., our guess for $\lambda_1$ converges from below.
\item As we discussed in class, one reason to discuss convergence of the entire subspace is that it is insensitive to gaps (or the lack thereof) between the first $\ell$ eigenvalues. If we instead assume $\lambda_1 > \lambda_2 > \cdots > \lambda_\ell > \lambda_{\ell+1}\geq \cdots \geq \lambda_n \geq 0,$ would we expect the columns of $V^{(k)}$ (the ON basis for our guess at the invariant subspace of interest at iteration $k$) to converge to individual eigenvectors (in an appropriate sense)? If so, what might be expect the asymptotic rates of convergence to be? (For this last part a convincing argument suffices, we do not need a formal proof.)
\end{enumerate}

\subsection*{Question 3 (a more challenging, ungraded problem):}
Let $A$ be a $n\times n$ matrix that is not diagonalizable, and whose eigenvalue of largest magnitude, denoted $\lambda_1,$ is associated with a Jordan block of size two. You may assume the rest of the eigenvalues ($\lambda_2,\ldots,\lambda_{n-1}$) are simple. This means that there exists a matrix $X$ such that 
\[
X^{-1}AX = \begin{bmatrix}\lambda_1 & 1 & \\ & \lambda_1 & \\ & & \Lambda \end{bmatrix}
\]
where $\Lambda$ is a diagonal matrix and $\|\Lambda\|_2<\lvert \lambda_1\rvert.$ Given essentially any initial guess, what, if anything, does the power method applied to $A$ converge to? If it does converge, at what rate does it do so? 

\end{document}