\documentclass[12pt]{article}
\usepackage[margin=1in]{geometry}
\usepackage[all]{xy}


\usepackage{amssymb, amsmath, amsthm,graphicx, paralist,algpseudocode,algorithm,cancel,url,color}
\usepackage{geometry}        
\geometry{letterpaper}    
\usepackage{graphicx}
\usepackage{enumitem}
\usepackage{cleveref}

\newenvironment{solution}[1][\it{Solution}]{\textbf{#1. } }{$\square$}

\newcommand{\ls}{\textsc{ls}}
\newcommand{\bvec}[1]{\mathbf{#1}}
\newcommand{\R}{\mathbb{R}}
\newcommand{\C}{\mathcal{C}}
\newcommand{\Rn}{\R^{n\times n}}
\newcommand{\Rmn}{\R^{m\times n}}
\newcommand{\Cn}{\C^{n\times n}}
\newcommand{\Cmn}{\C^{m\times n}}
\newcommand{\cO}{\mathcal{O}}
\DeclareMathOperator{\Tr}{Tr}
\DeclareMathOperator{\trace}{trace}
\DeclareMathOperator{\diag}{diag}
\DeclareMathOperator*{\argmin}{arg\,min}
\DeclareMathOperator*{\argmax}{arg\,max}

\begin{document}
\noindent
\textsc{\Large Numerical analysis: Project 01} \\ 
\noindent \textbf{Students}:
Cedric Orton-Urbina (ceo29),
Aditya Syam (as2839),
Pratyush Sudhakar (ps2245)
\\
\noindent
\textbf{Due Date}: 11th March, 2024
\\

\hrule

\section*{Question 1.}
Show that if $z$ solves 
\[
    \min_{z} \|A(:,1:k)z - b\|_2^2 
\]
then 
\[
    \|A\begin{bmatrix}z \\ 0\end{bmatrix}-b\|_2 \leq \|R_{22}\|_2\|x_{\ls}\|_2 + \|b-Ax_{\ls}\|_2,
\]
where $x_{\ls}$ solves 
\[
    \min_{x} \|Ax - b\|_2^2. 
\]
In other words, the residual from the sparse solution cannot be much larger than the optimal residual. 

\subsection*{Answer 1.}

To begin, let $A(:, 1:k) = A_k$ and $A(:, k+1:n) = A_m$ for simplicity. We begin identifying the meaning behind each term in the above question. 

\begin{enumerate}
    \item $z$ is a vector containing the coefficients that best approximate $b$ using the columns of $A_k$.
          
    \item $A\begin{bmatrix}z \\ 0\end{bmatrix}-b$ is the spare solution $z$ extended to the full column space of $A$
          
    \item $\|b-Ax_{\ls}\|_2$ is the norm of the minimum achievable residual of the least squares problem in this question.
          
    \item $R_{22}$ represents the portion of $A$ not involved in the initial $k$ column solution.
\end{enumerate}

We begin by noting that the residual $\|A\begin{bmatrix}z \\ 0\end{bmatrix}-b\|_2$ contains contributions from two distinct parts of $A$. This first contribution is attributed to the first $k$ columns of $A$, which are scaled by $z$. The residual $\| b - A_k z\|_2$ is bounded below by the minimal residual $r_{min} = \|b-Ax_{\ls}\|_2$. The case where these two residuals would be equal corresponds to an $A$ where all other columns are $0$ aside from the first $k$. In other words, these residuals are equal when $A_m = 0$, but otherwise, 

$$\| b - A_k z\|_2 \geq r_{min}$$

We now rewrite the original inequality as follows:

$$\|A\begin{bmatrix}z \\ 0\end{bmatrix}-b\|_2 - \|b-Ax_{\ls}\|_2 \leq \|R_{22}\|_2\|x_{\ls}\|_2$$

Therefore, we must quantify the increased accuracy in our least squares solution when $A_m$ is included in our minimization. q


\section*{Question 2.}
Show that there exists a matrix $B$ such that
\[
    \|A - A(:,1:k)B\|_2 = \|R_{22}\|_2
\]
and 
\[
    \|A(:,1:k) - A(:,1:k)B(:,1:k)\|_2 = 0.
\]
The second condition ensures that $k$ columns of $A$ are exactly represented by the rank-$k$ factorization $A(:,1:k)B.$

\subsection*{Answer 2.}

\section*{Question 3.}
Show that
\[
    \|R_{22}\|_2\leq \sigma_1(A)
\]
and that for any $m,n,$ and $k$ with $m\geq n$ and $k < \min(m,n)$ there exists a matrix $A$ such that
\[
    \|R_{22}\|_2 = \sigma_1(A).
\]
This shows that while $R_{22}$ is bounded in size by the scale of $A,$ there are examples where it is that large.

\subsection*{Answer 3.}

We begin by noting that $\sigma_1(A)$ is the largest singular value of $A$, and is also equal to the 2-norm of A. \\

Since $R_{22}$ is a partition of the QR factorization of A, and QR factorization does not increase any spectral properties of $A$ (magnitude of singular values), $\|R_{22}\|_2$ can at most be $\sigma_1$. This would be the case if $R_{22}$ contained the singular value $\sigma_1$, otherwise,  $\|R_{22}\|_2$ will be less than $\sigma_1$. Thus,

$$\|R_{22}\|_2\leq \sigma_1(A)$$ \\

To show there exists $A$ such that $\|R_{22}\|_2 = \sigma_1(A)$, consider a matrix $A\in \Rmn$ with entries only along the diagonal ($A_{ii}$ for $i \leq min(m,n)$). We then place the largest of these entries ($\sigma_1$ in the case of a diagonal matrix) in the $(k+1)^{th}$ diagonal of $A$. Since the QR decomposition of a diagonal matrix $A$ with excess rows or columns is just the identity matrix times $A$, we know that the largest singular value $\sigma_1$ will be in the $R_{22}$ partition of the QR factorization of $A$. The 2-norm of $R_{22}$ will be equal to it's largest singular value, which in this case, is the same largest singular value of $A$. Thus, there exists $A \in \Rmn$ such that  

$$\|R_{22}\|_2\ = \sigma_1(A)$$

\section*{Question 4.}
\begin{equation}
    \label{eqn:pivotQRk}
    A\begin{bmatrix}\Pi_1 & \Pi_2\end{bmatrix} = \begin{bmatrix} Q_1 & Q_2\end{bmatrix} \begin{bmatrix} R_{11} & R_{12} \\ & R_{22}\end{bmatrix}
\end{equation} \\

Show that if we have a factorization of the form~\cref{eqn:pivotQRk} then there exists a matrix $B$ such that
\[
    \|A - A(:,\C)B\|_2 = \|R_{22}\|_2
\]
and
\[
    \|A(:,\C) - A(:,\C)B(:,\C)\|_2 = 0.
\]
Here we are just rewriting the earlier result using the permutations. A similar thing can be done for the least-squares problem. 

\subsection*{Answer 4.}

\section*{Question 5.}
The greedy heuristic we will follow is to remove the largest column from $R_{22}$ at each step. Specifically, lets say we have completed $k$ steps of our process have have the partial factorization
\begin{equation}
    \label{eqn:PHQRk}
    Q^{(k)}\cdots Q^{(1)}A\Pi^{(1)}\cdots\Pi^{(k)} = \begin{bmatrix} R_{11} & R_{12} \\ & M\end{bmatrix},
\end{equation}
where $R_{11}\in\R^{k\times k}$ is upper triangular and $M\in\R^{(m-k)\times (n-k)}$ is dense (because we have not yet reduced it to upper triangular form). \\

Show that if we continue the factorization in~\cref{eqn:PHQRk} without any additional permutations then $\|R_{22}\|_2 = \|M\|_2.$

\subsection*{Answer 5.}

\section*{Question 6.}
Show that $\Pi_1$ in~\cref{eqn:pivotQRk} only depends on $\Pi^{(1)}$ through $\Pi^{(k)}.$ In other words, if we just want/need $\C$ we can stop the factorization after $k$ steps. We can also recover the matrix $B$ in our low-rank factorization without proceeding further.

\subsection*{Answer 6.}




\end{document}
