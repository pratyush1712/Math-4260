
\documentclass[11pt,onecolumn]{article}
\usepackage{amssymb, amsmath, amsthm,graphicx, paralist,algpseudocode,algorithm,cancel,url,color}
\usepackage{sectsty}
\usepackage{fancyvrb}
\usepackage{mathrsfs}
\usepackage{multirow}
\usepackage{hhline}
\usepackage{booktabs}
\usepackage[table]{xcolor}
\usepackage{tikz}
% \usepackage[framed,numbered,autolinebreaks,useliterate]{mcode}
\usepackage{listings}
\usepackage{enumitem}

\newcommand{\bvec}[1]{\mathbf{#1}}
\newcommand{\R}{\mathbb{R}}
\newcommand{\C}{\mathbb{C}}
\newcommand{\Rn}{\R^{n\times n}}
\newcommand{\Rmn}{\R^{m\times n}}
\newcommand{\Cn}{\C^{n\times n}}
\newcommand{\Cmn}{\C^{m\times n}}
\newcommand{\cO}{\mathcal{O}}
\DeclareMathOperator{\Tr}{Tr}
\DeclareMathOperator{\trace}{trace}
\DeclareMathOperator{\diag}{diag}
\DeclareMathOperator{\vspan}{span}
\sectionfont{\Large\sc}
\subsectionfont{\sc}
\usepackage[margin=1 in]{geometry}
\begin{document}
\noindent
\textsc{\Large Numerical analysis: Homework 6}\\
Pratyush Sudhakar
\\
\vspace{3pt}
\hrule

\section*{Question:}
The question relates to the QR factorization. Let \( A \) be a matrix of dimension \( m \times n \). Assume that \( A \) has linearly independent columns. Given two QR factorizations of the form \( A = QR \) and \( A = ST \), show that \( Q = S \) and \( R = T \). Essentially, we want to show that the QR factorization is unique.
\\
\vspace{1pt}
\\
\textbf{Solution:}
We can express \( Q \) as \( Q = [q_1 \, q_2 \, \ldots \, q_n] \). Similarly, \( S = [s_1 \, s_2 \, \ldots \, s_n] \). After expressing the matrices in terms of their columns, we observe that \( Q^T Q = I_n = S^T S \). This holds because both \( Q \) and \( S \) have orthonormal columns. Hence, we can simply show that \( S = Q \), because this then gives \( T = S^T A = Q^T A = R \).
\\
\vspace{1pt}
\\
Now, since \( S^T S = I_n \), the equation \( QR = S^T \) gives \( S^T Q = T R^{-1} \), and this matrix is expressed as
\[
S^T Q = T R^{-1} = [t_{ij}]
\]
This matrix is clearly upper triangular with positive diagonal elements (since this is true for both \( R \) and \( T \)). As a result, \( t_{ii} > 0 \) for each \( i \) and \( t_{ij} = 0 \) if \( i > j \). On the other hand, the \( (i, j) \)-entry of \( S^T Q \) is \( s_i^T q_j = s_i \cdot q_j \), so we have \( s_i \cdot q_j = t_{ij} \) for all \( i \) and \( j \). But each \( q_j \) is in \( \text{span} \{s_1, s_2, \ldots, s_n\} \) because \( Q = S(T R^{-1}) \). Hence the expansion theorem gives
\[
q_j = (s_1 \cdot q_j) s_1 + (s_2 \cdot q_j) s_2 + \ldots + (s_n \cdot q_j) s_n = t_{1j} s_1 + t_{2j} s_2 + \ldots + t_{jj} s_j
\]
because \( s_i \cdot q_j = t_{ij} = 0 \) if \( i > j \).
\\
Expanding the terms, the first few equations we get are
\[
\begin{aligned}
q_1 &= t_{11} s_1 \\
q_2 &= t_{12} s_1 + t_{22} s_2 \\
q_3 &= t_{13} s_1 + t_{23} s_2 + t_{33} s_3 \\
q_4 &= t_{14} s_1 + t_{24} s_2 + t_{34} s_3 + t_{44} s_4 \\
&\vdots
\end{aligned}
\]
The first one gives \( 1 = \|q_1\| = \|t_{11} s_1\| = |t_{11}| \|s_1\| = t_{11} \). This means \( q_1 = s_1 \).
\\
Now, we have \( t_{12} = s_1 \cdot q_2 = q_1 \cdot q_2 = 0 \), so the second equation is \( q_2 = t_{22} s_2 \). Following a similar logic, we get \( q_2 = s_2 \), resulting in \( t_{13} = 0 \) and \( t_{23} = 0 \). This then gives us \( q_3 = t_{33} s_3 \) and \( q_3 = s_3 \).
\\
\vspace{1pt}
\\
We can take this induction forward to get \( q_i = s_i \) for all \( i \). This means that \( Q = S \), which is what we needed to show. This also implies that \( T = S^T A = Q^T A = R \). Essentially, \( QR = S^T \) and the QR factorization is unique. This concludes the proof.

\end{document}