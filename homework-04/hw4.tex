
\documentclass[11pt,onecolumn]{article}
\usepackage{amssymb, amsmath, amsthm,graphicx, paralist,algpseudocode,algorithm,cancel,url,color}
\usepackage{sectsty}
\usepackage{fancyvrb}
\usepackage{mathrsfs}
\usepackage{multirow}
\usepackage{hhline}
\usepackage{booktabs}
\usepackage[table]{xcolor}
\usepackage{tikz}
% \usepackage[framed,numbered,autolinebreaks,useliterate]{mcode}
\usepackage{listings}
\usepackage{enumitem}

\newcommand{\bvec}[1]{\mathbf{#1}}
\newcommand{\R}{\mathbb{R}}
\newcommand{\C}{\mathbb{C}}
\newcommand{\Rn}{\R^{n\times n}}
\newcommand{\Rmn}{\R^{m\times n}}
\newcommand{\Cn}{\C^{n\times n}}
\newcommand{\Cmn}{\C^{m\times n}}
\newcommand{\cO}{\mathcal{O}}
\DeclareMathOperator{\Tr}{Tr}
\DeclareMathOperator{\trace}{trace}
\DeclareMathOperator{\diag}{diag}
\DeclareMathOperator{\vspan}{span}
\sectionfont{\Large\sc}
\subsectionfont{\sc}
\usepackage[margin=1 in]{geometry}
\begin{document}
\noindent
\textsc{\Large Numerical analysis: Homework 4}\\
Pratyush Sudhakar\\

\subsection*{Question 1:}
Implement Newton's method for root finding. For each of the following compute a root of the function and illustrate the order of convergence.
\\
\vspace{0pt}
\\
\textbf{Solution:} 
Newton's method:
\[
x_{k+1} = x_k - \frac{f(x_k)}{f'(x_k)}
\]
where \( x_k \) is the current approximation, \( f(x_k) \) is the function value at \( x_k \), and \( f'(x_k) \) is the derivative of the function at \( x_k \).


\begin{enumerate}[label=(\alph*)]
\item $f(x) = x^2$\\
\textbf{Solution:} 
\[
f(x) = x^2, \quad f'(x) = 2x
\]
\[
x_{k+1} = x_k - \frac{x_k^2}{2x_k} = \frac{x_k}{2}
\]

The root of \( f(x) = x^2 \) is \( x^* = 0 \).

Let \( e_k = x^* - x_k = -x_k \). Then,
\[
|e_{k+1}| = \left| x^* - x_{k+1} \right| = \left| 0 - \frac{x_k}{2} \right| = \frac{|x_k|}{2} = \frac{|e_k|}{2}
\]
Since \( |e_{k+1}| = \rho |e_k| \) with \( \rho = \frac{1}{2} \), the method exhibits linear convergence with rate \( \rho = \frac{1}{2} \).

\item $f(x) = \sin{x} + x^3$ \\
\textbf{Solution:}
\[
f(x) = \sin{x} + x^3, \quad f'(x) = \cos{x} + 3x^2
\]

The root of \( f(x) = \sin{x} + x^3 \) can be found numerically. One root is \( x^* \approx 0 \).
Using Taylor expansion around \( x^* = 0 \):
\[
\sin{x} \approx x - \frac{x^3}{6} + O(x^5)
\]
\[
\cos{x} \approx 1 - \frac{x^2}{2} + O(x^4)
\]

At each step of Newton's method:
\[
x_{k+1} = x_k - \frac{\sin{x_k} + x_k^3}{\cos{x_k} + 3x_k^2}
\]

Expanding \( f(x_k) \) and \( f'(x_k) \) around \( x^* = 0 \):
\[
f(x_k) \approx x_k - \frac{x_k^3}{6} + x_k^3 = x_k + \frac{5}{6} x_k^3
\]
\[
f'(x_k) \approx 1 - \frac{x_k^2}{2} + 3x_k^2 = 1 + \frac{5}{2} x_k^2
\]

Thus,
\[
x_{k+1} \approx x_k - \frac{x_k + \frac{5}{6} x_k^3}{1 + \frac{5}{2} x_k^2}
\]

Approximating the denominator using a Taylor expansion:
\[
\frac{1}{1 + \frac{5}{2} x_k^2} \approx 1 - \frac{5}{2} x_k^2
\]

Hence,
\[
x_{k+1} \approx x_k - \left( x_k + \frac{5}{6} x_k^3 \right) \left( 1 - \frac{5}{2} x_k^2 \right)
\]
\[
x_{k+1} \approx x_k - \left( x_k + \frac{5}{6} x_k^3 - \frac{5}{2} x_k^3 \right)
\]
\[
x_{k+1} \approx x_k - \left( x_k + \frac{5}{3} x_k^3 \right)
\]
\[
x_{k+1} \approx x_k - x_k - \frac{5}{3} x_k^3
\]
\[
x_{k+1} \approx -\frac{5}{3} x_k^3
\]

Thus, the quadratic term vanishes and the convergence is cubic. Therefore,
\[
|e_{k+1}| \approx C |e_k|^3
\]



\item $f(x) = \sin{\frac{1}{x}}$ for $x\neq 0$ \\
\textbf{Solution:}
\[
f(x) = \sin{\frac{1}{x}}, \quad f'(x) = -\frac{\cos{\frac{1}{x}}}{x^2}
\]

\paragraph{Convergence Analysis:}
The function \( f(x) = \sin{\frac{1}{x}} \) has a root at \( x \to \infty \), which is not practical for numerical methods. Newton's method may not converge due to the oscillatory and discontinuous nature of the function around the root.

\paragraph{Order of Convergence:}
Since the function is highly oscillatory and discontinuous at the root, the order of convergence is difficult to determine and is likely to be non-standard.


\end{enumerate}

\subsection*{Question 2:}
Show that given any initial guess the Jacobi method for solving $Ax=b $converges for any strictly diagonally dominant matrix $A$ (i.e., $\lvert a_{ii} \rvert > \sum_{j\neq i} \lvert a_{ij} \rvert$ for $i=1,\ldots,n$).
\\
\vspace{0pt}
\\
\textbf{Solution:}
Consider the linear system \(Ax = b\), where \(A\) is a strictly diagonally dominant matrix.
The Jacobi method for solving \(Ax = b\) can be written in matrix form as:
\[
x^{(k+1)} = D^{-1}(b - (L + U)x^{(k)})
\]
where \(A = D + L + U\), with \(D\) being the diagonal part of \(A\), \(L\) the strictly lower triangular part, and \(U\) the strictly upper triangular part.
Rewriting the Jacobi iteration:
\[
x^{(k+1)} = D^{-1}(b - (L + U)x^{(k)}) = D^{-1}b - D^{-1}(L + U)x^{(k)}
\]
\\
Defining \(T_J = -D^{-1}(L + U)\), the iteration can be expressed as:
\[
x^{(k+1)} = T_J x^{(k)} + D^{-1}b
\]
\\
The Jacobi method converges if and only if the spectral radius \(\rho(T_J)\) of the iteration matrix \(T_J\) is less than 1, i.e., \(\rho(T_J) < 1\).
\\
Next, we apply the Gershgorin circle theorem. For the matrix \(T_J = -D^{-1}(L + U)\):
\[
(T_J)_{ii} = 0 \quad \text{(since the diagonal elements are zero)}
\]
\[
(T_J)_{ij} = -\frac{a_{ij}}{a_{ii}} \quad \text{for } i \neq j
\]
\\
Each Gershgorin disc for \(T_J\) is centered at 0 with radius:
\[
R_i = \sum_{j \neq i} \left| (T_J)_{ij} \right| = \sum_{j \neq i} \left| -\frac{a_{ij}}{a_{ii}} \right| = \frac{1}{|a_{ii}|} \sum_{j \neq i} |a_{ij}|
\]
\\
Since \(A\) is strictly diagonally dominant:
\[
|a_{ii}| > \sum_{j \neq i} |a_{ij}|
\]
\\
Hence,
\[
1 > \sum_{j \neq i} \left| \frac{a_{ij}}{a_{ii}} \right|
\]
\\
This implies:
\[
\sum_{j \neq i} \left| (T_J)_{ij} \right| < 1
\]
\\
According to the Gershgorin circle theorem, all eigenvalues of \(T_J\) lie within the union of the Gershgorin discs, each centered at 0 with radius less than 1. Hence, all eigenvalues of \(T_J\) lie strictly within the unit circle in the complex plane:
\[
\rho(T_J) < 1
\]
\\
Since the spectral radius of \(T_J\) is less than 1, the Jacobi method converges for any initial guess \(x^{(0)}\).
Hence, proved.

\subsection*{Question 3:}
Prove that if $\nabla f(x) = 0$ but the Hessian $\nabla^2f(x)$ is indefinite (i.e., has positive and negative eigenvalues), then there is a direction we can move from $x$ that decreases the function value provided we take a small enough step. (I.e., show that $x$ is not a local minimizer and that we can make progress when running an optimization scheme using so-called directions of negative curvature.)

\vspace*{12pt}

\noindent
\textbf{Solution:}
\\
\\
\textbf{NOTE VERIFIED NOT VERY SURE ABOUT THE PROOF}
\\
\\
Given that \(\nabla f(x) = 0\) and the Hessian \(\nabla^2 f(x)\) is indefinite, we need to show that there exists a direction in which moving from \(x\) decreases the function value.
\\
The second-order Taylor expansion of \(f\) around a point \(x\) is:
\[
f(x + p) \approx f(x) + \nabla f(x)^T p + \frac{1}{2} p^T \nabla^2 f(x) p
\]
\\
Since \(\nabla f(x) = 0\), this simplifies to:
\[
f(x + p) \approx f(x) + \frac{1}{2} p^T \nabla^2 f(x) p
\]
\\
The Hessian \(\nabla^2 f(x)\) being indefinite means it has both positive and negative eigenvalues. Therefore, there exists a direction \(p\) such that:
\[
p^T \nabla^2 f(x) p < 0
\]
\\
Consider the Rayleigh quotient:
\[
R(p) = \frac{p^T \nabla^2 f(x) p}{p^T p}
\]
The Rayleigh quotient achieves its minimum and maximum values at the eigenvalues of \(\nabla^2 f(x)\). Since \(\nabla^2 f(x)\) is indefinite, its eigenvalues include both positive and negative values, ensuring that there is some direction \(p\) for which \(R(p) < 0\).
\\
Let \(p\) be an eigenvector corresponding to a negative eigenvalue \(\lambda\) of \(\nabla^2 f(x)\). Then:
\[
\nabla^2 f(x) p = \lambda p
\]
and
\[
p^T \nabla^2 f(x) p = \lambda p^T p
\]
\\
Since \(\lambda < 0\):
\[
p^T \nabla^2 f(x) p < 0
\]
\\
By choosing \(p\) in the direction of this eigenvector, we can ensure that the second-order term in the Taylor expansion is negative:
\[
f(x + p) \approx f(x) + \frac{1}{2} p^T \nabla^2 f(x) p < f(x)
\]
\\
Thus, for a sufficiently small step size \(\alpha > 0\), moving in the direction \(p\) will decrease the function value:
\[
f(x + \alpha p) < f(x)
\]
\\
This implies that \(x\) is not a local minimizer, as there exists a direction \(p\) in which the function value decreases. Hence, we can make progress in an optimization scheme by moving in directions of negative curvature.

\end{document}